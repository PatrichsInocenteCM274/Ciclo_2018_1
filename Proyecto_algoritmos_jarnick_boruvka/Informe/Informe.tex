\documentclass[a4paper]{article}

%% Language and font encodings
\usepackage[english]{babel}
\usepackage[utf8x]{inputenc}
\usepackage[T1]{fontenc}
\usepackage{wrapfig, blindtext}

%% Sets page size and margins
\usepackage[a4paper,top=3cm,bottom=2cm,left=3cm,right=3cm,marginparwidth=1.75cm]{geometry}

%% Useful packages
\usepackage{amsmath}
\usepackage{graphicx}
\usepackage[colorinlistoftodos]{todonotes}
\usepackage[colorlinks=true, allcolors=blue]{hyperref}

%Caratula
\begin{document}
\begin{titlepage}
\begin{center}
\vspace*{-0.4in}

{\fontsize{12}{30}\bf \selectfont UNIVERSIDAD NACIONAL DE INGENIERIA\\}

{\fontsize{12}{40}\bf \selectfont FACULTAD DE CIENCIAS\\}
\vspace*{0.15in} ESCUELA PROFESIONAL DE CIENCIAS DE LA COMPUTACI\'ON\\
\vspace*{0.08in}
\vspace*{0.15in} ESCUELA PROFESIONAL DE MATEM\'ATICA\\
\vspace*{0.2in}
\begin{figure}[htb]
\begin{center}
\includegraphics[width=4.5cm,height=6.5cm]{UNI.png}
\end{center}
\end{figure}
\begin{Large}
\textbf{ALGORITMOS DE JARNICK Y BORUVKA\\}
\end{Large}
\vspace*{0.2in}

\begin{large}
{\bf T\'itulo del Trabajo\\}
\vspace*{0.1in}
{\fontsize{12}{13}\selectfont 
Encontrando el Arbol generador minimal de un grafo, mediante la implementaci\'on de los algoritmos de Jarnick y Boruvka en lenguaje C\\}
\end{large}
\vspace*{0.3in}

\begin{large}
{\bf Autores} 
\vspace*{0.1in}
\\Rojas D\'iaz, Ivan\\
Pantoja, Fabio \\Alarco, Andrew\\
Inocente Valle, Patrichs
\end{large}

\vspace*{0.4in}
\begin{large}
{\bf Profesor} 
\vspace*{0.1in}
\\Echegaray Castillo, William Carlos
\end{large}

\end{center}
\begin{center}
\begin{large}
\vspace*{0.5in}
Lima - Peru\\
{\bf JUNIO 2018}
\end{large}
\end{center}
\end{titlepage}

\pagebreak
\tableofcontents
\pagebreak

\section{Resumen}
\subsection{ARBOL GENERADOR MINIMAL}
\subsubsection{Conceptos Generales}
\section{Implementaci\'on}
\section{Pruebas y Resultados}
\section{Conclusi\'on}


\end{document}
